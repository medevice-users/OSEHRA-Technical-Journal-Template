\documentclass{OSEHRAArticle}

\usepackage[dvips]{graphicx}
\usepackage{color}
\usepackage{listings}
\usepackage{verbatim}
\usepackage{textcomp}

\definecolor{listcomment}{rgb}{0.0,0.5,0.0}
\definecolor{listkeyword}{rgb}{0.0,0.0,0.5}
\definecolor{listnumbers}{gray}{0.65}
\definecolor{listlightgray}{gray}{0.955}
\definecolor{listwhite}{gray}{1.0}


%%%%%%%%%%%%%%%%%%%%%%%%%%%%%%%%%%%%%%%%%%%%%%%%%%%%%%%%%%%%%%%%%%
%
%  hyperref should be the last package to be loaded.
%
%%%%%%%%%%%%%%%%%%%%%%%%%%%%%%%%%%%%%%%%%%%%%%%%%%%%%%%%%%%%%%%%%%
\usepackage[dvips,
bookmarks,
bookmarksopen,
backref,
colorlinks,linkcolor={blue},citecolor={blue},urlcolor={blue},
]{hyperref}


\newcommand{\lstlistingwithnumber}[3]{
\begin{center}
\lstinputlisting[linerange={#1-#2},firstnumber=#1,numbers=left]{#3}
\end{center}
}


\title{Converter from Cache to GT.M}


%
% NOTE: This is the last number of the "handle" URL that
% The OSEHRA Journal assigns to your paper as part of the
% submission process. Please replace the number "1" with
% the actual handle number that you get assigned.
%
\newcommand{\OTJhandlerIDnumber}{17}

\lstset{frame = tb,
       framerule = 0.25pt,
       float,
       fontadjust,
       backgroundcolor={\color{listlightgray}},
       basicstyle = {\ttfamily\footnotesize},
       keywordstyle = {\ttfamily\color{listkeyword}\textbf},
       identifierstyle = {\ttfamily},
       commentstyle = {\ttfamily\color{listcomment}\textit},
       stringstyle = {\ttfamily},
       showstringspaces = false,
       showtabs = false,
       numbers = left,
       numbersep = 6pt,
       numberstyle={\ttfamily\color{listnumbers}},
       tabsize = 2,
       language=[ANSI]C++,
       floatplacement=!h
       }

\release{1.10}

\author{Rick Spivey}
\authoraddress{Department of Veterans Affairs}

\begin{document}


%
% Add hyperlink to the web location and license of the paper.
% The argument of this command is the handler identifier given
% by the OSEHRA Journal to this paper.
%
\OTJhandlefooter{\OTJhandlerIDnumber}


\ifpdf
\else
   %
   % Commands for including Graphics when using latex
   %
   \DeclareGraphicsExtensions{.eps,.jpg,.gif,.tiff,.bmp,.png}
   \DeclareGraphicsRule{.jpg}{eps}{.jpg.bb}{`convert #1 eps:-}
   \DeclareGraphicsRule{.gif}{eps}{.gif.bb}{`convert #1 eps:-}
   \DeclareGraphicsRule{.tiff}{eps}{.tiff.bb}{`convert #1 eps:-}
   \DeclareGraphicsRule{.bmp}{eps}{.bmp.bb}{`convert #1 eps:-}
   \DeclareGraphicsRule{.png}{eps}{.png.bb}{`convert #1 eps:-}
\fi


\maketitle


\ifhtml
\chapter*{Front Matter\label{front}}
\fi


\begin{abstract}
\noindent
This article describes a M Routine that can take other M routines from the
Intersystems Cache environment and convert them to a format suitable for being
used in GT.M.
\end{abstract}

\tableofcontents

\section{Introduction}
There are certain differences between the format of M routines that are used in
Intersystems Cache and GT.M. The M routine in this contribution is a helpful
tool for converting M routines from an Intersystems Cache environment to a
format that can be used in GT.M.

\section{How to use the code}

\subsection{Platform}

This routine was written for Cache that runs on a Windows PC.

\subsection{Running the code}

In order to run this code, please execute the following steps:

\begin{itemize}
\item Create a GTM folder in the root folder of the namespace that this routine runs in.
\item Run LTUGTM to specify which routines to convert.
\item Run ALL LTUGTM to have the utility convert all routines in the namespace from A to ZZZZZZZ.
\end{itemize}


\section{Source Code Repository}

The code is being hosted in the following Git repository:

\begin{itemize}
\item \url{https://github.com/OSEHR/CacheToGTM}
\end{itemize}


\section{Portability Suggestions}

This routine can probably be modified to run in a Cache VMS environment by
asking the user for a directory to put the .m files into and replacing the
literal "GTM/" with that variable.


\section{Source Code}

Given that this is a very compact code base, we can list it here for your convenience.

\lstlistingwithnumber{1}{71}{LTUGTM.RO}



%%%%%%%%%%%%%%%%%%%%%%%%%%%%%%%%%%%%%%%%%
%
%  Insert the bibliography using BibTeX
%
%%%%%%%%%%%%%%%%%%%%%%%%%%%%%%%%%%%%%%%%%

\bibliographystyle{plain}
\bibliography{OSEHRAJournal}


\end{document}
