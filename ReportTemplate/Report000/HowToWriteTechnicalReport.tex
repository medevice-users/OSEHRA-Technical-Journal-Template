\documentclass{OSEHRAArticle}

\usepackage[dvips]{graphicx}
\usepackage{color}
\usepackage{listings}
\usepackage{verbatim}
\usepackage{textcomp}

\definecolor{listcomment}{rgb}{0.0,0.5,0.0}
\definecolor{listkeyword}{rgb}{0.0,0.0,0.5}
\definecolor{listnumbers}{gray}{0.65}
\definecolor{listlightgray}{gray}{0.955}
\definecolor{listwhite}{gray}{1.0}


%%%%%%%%%%%%%%%%%%%%%%%%%%%%%%%%%%%%%%%%%%%%%%%%%%%%%%%%%%%%%%%%%%
%
%  hyperref should be the last package to be loaded.
%
%%%%%%%%%%%%%%%%%%%%%%%%%%%%%%%%%%%%%%%%%%%%%%%%%%%%%%%%%%%%%%%%%%
\usepackage[dvips,
bookmarks,
bookmarksopen,
backref,
colorlinks,linkcolor={blue},citecolor={blue},urlcolor={blue},
]{hyperref}


\newcommand{\lstlistingwithnumber}[3]{
\begin{center}
\lstinputlisting[linerange={#1-#2},firstnumber=#1,numbers=left]{#3}
\end{center}
}


\title{How to Write an OSEHRA Technical Report}


%
% NOTE: This is the last number of the "handle" URL that
% The OSEHRA Journal assigns to your paper as part of the
% submission process. Please replace the number "1" with
% the actual handle number that you get assigned.
%
\newcommand{\OTJhandlerIDnumber}{2}

\lstset{frame = tb,
       framerule = 0.25pt,
       float,
       fontadjust,
       backgroundcolor={\color{listlightgray}},
       basicstyle = {\ttfamily\footnotesize},
       keywordstyle = {\ttfamily\color{listkeyword}\textbf},
       identifierstyle = {\ttfamily},
       commentstyle = {\ttfamily\color{listcomment}\textit},
       stringstyle = {\ttfamily},
       showstringspaces = false,
       showtabs = false,
       numbers = left,
       numbersep = 6pt,
       numberstyle={\ttfamily\color{listnumbers}},
       tabsize = 2,
       language=[ANSI]C++,
       floatplacement=!h
       }

\release{1.10}

\author{OSHERA Agent}

\begin{document}


%
% Add hyperlink to the web location and license of the paper.
% The argument of this command is the handler identifier given
% by the OSEHRA Journal to this paper.
%
\OTJhandlefooter{\OTJhandlerIDnumber}


\ifpdf
\else
   %
   % Commands for including Graphics when using latex
   %
   \DeclareGraphicsExtensions{.eps,.jpg,.gif,.tiff,.bmp,.png}
   \DeclareGraphicsRule{.jpg}{eps}{.jpg.bb}{`convert #1 eps:-}
   \DeclareGraphicsRule{.gif}{eps}{.gif.bb}{`convert #1 eps:-}
   \DeclareGraphicsRule{.tiff}{eps}{.tiff.bb}{`convert #1 eps:-}
   \DeclareGraphicsRule{.bmp}{eps}{.bmp.bb}{`convert #1 eps:-}
   \DeclareGraphicsRule{.png}{eps}{.png.bb}{`convert #1 eps:-}
\fi


\maketitle


\ifhtml
\chapter*{Front Matter\label{front}}
\fi


\begin{abstract}
\noindent
This paper describes the basic guidelines for preparing technical
reports to be contributed to the OSEHRA Technical Journal.

\begin{center}
\url{http://code.osehra.org/journal/}
\end{center}

The purpose of the Technical Journal is to foster innovation in the OSEHRA
community and to drive this innovations towards a mature state, at which point
it can be incorporated into the Open Source EHR software platform.
\end{abstract}

\tableofcontents

\section{Introduction}

Technical reports for the OSEHRA Technical Journal are written by software
developers and are intended for software developers. As such, they are required
to include the source code of any components that is the subject of the report,
and to include as well the unit tests and required data and parameters needed
to run such tests.

A successful OSEHRA Technical Report should be such that any reader can download
the report and its acompanying materials and with minimal effort verify that the
software works and produce useful results.

\section{Code Walkthrough}

In this hypothetical example, we do a walk throught the source code of the
contribution with the purpose of highlighting some of the key features in this
source code piece.

First we say something smart about the first five lines:

\lstlistingwithnumber{1}{5}{PROGRAM1.m}

That is followed by an explanation of why the next three lines are required:

\lstlistingwithnumber{6}{9}{PROGRAM1.m}

and conclude with a commentary of the last lines:

\lstlistingwithnumber{10}{14}{PROGRAM1.m}



\section{How to use the code}

In this typical section we explain how a typical OSEHRA developer can take
advantage of the contribution described in this technical report.


\begin{verbatim}
mumps  PROGRAM1.m
\end{verbatim}


\section{Preparation}

A typical contribution to the OSEHRA Technical Journal should include the following elements

\begin{itemize}
\item PDF Document with a Technical Report
\item PDF Document with the Primary Reviewer Checklist
\item Source code of the contribution
\item Unite Testing for the contributed source code
\end{itemize}

During the process of uploading your contribution to the OSEHRA Technical
Journal, you will be asked to upload each one of these items.

More detailed instructions are available in the OSEHRA Web site at:

\url{http://www.osehra.org/document/code-contributions-and-review-procedures}


\section{Conclusions}

In the final section we discuss potential improvements to the code, and what
other uses it could have.


%%%%%%%%%%%%%%%%%%%%%%%%%%%%%%%%%%%%%%%%%
%
%  Insert the bibliography using BibTeX
%
%%%%%%%%%%%%%%%%%%%%%%%%%%%%%%%%%%%%%%%%%

\bibliographystyle{plain}
\bibliography{OSEHRAJournal}


\end{document}
