\documentclass{OSEHRAArticle}

\usepackage[dvips]{graphicx}
\usepackage{color}
\usepackage{listings}
\usepackage{verbatim}
\usepackage{textcomp}

\definecolor{listcomment}{rgb}{0.0,0.5,0.0}
\definecolor{listkeyword}{rgb}{0.0,0.0,0.5}
\definecolor{listnumbers}{gray}{0.65}
\definecolor{listlightgray}{gray}{0.955}
\definecolor{listwhite}{gray}{1.0}


%%%%%%%%%%%%%%%%%%%%%%%%%%%%%%%%%%%%%%%%%%%%%%%%%%%%%%%%%%%%%%%%%%
%
%  hyperref should be the last package to be loaded.
%
%%%%%%%%%%%%%%%%%%%%%%%%%%%%%%%%%%%%%%%%%%%%%%%%%%%%%%%%%%%%%%%%%%
\usepackage[dvips,
bookmarks,
bookmarksopen,
backref,
colorlinks,linkcolor={blue},citecolor={blue},urlcolor={blue},
]{hyperref}


\newcommand{\lstlistingwithnumber}[3]{
\begin{center}
\lstinputlisting[linerange={#1-#2},firstnumber=#1,numbers=left]{#3}
\end{center}
}


\title{Appointment Reminder Card\\R1AC Print Postcard}


%
% NOTE: This is the last number of the "handle" URL that
% The OSEHRA Journal assigns to your paper as part of the
% submission process. Please replace the number "1" with
% the actual handle number that you get assigned.
%
\newcommand{\OTJhandlerIDnumber}{5}

\lstset{frame = tb,
       framerule = 0.25pt,
       float,
       fontadjust,
       backgroundcolor={\color{listlightgray}},
       basicstyle = {\ttfamily\footnotesize},
       keywordstyle = {\ttfamily\color{listkeyword}\textbf},
       identifierstyle = {\ttfamily},
       commentstyle = {\ttfamily\color{listcomment}\textit},
       stringstyle = {\ttfamily},
       showstringspaces = false,
       showtabs = false,
       numbers = left,
       numbersep = 6pt,
       numberstyle={\ttfamily\color{listnumbers}},
       tabsize = 2,
       language=[ANSI]C++,
       floatplacement=!h
       }

\release{1.10}

\author{Don Donati and Kevin DeZorzi}

\begin{document}


%
% Add hyperlink to the web location and license of the paper.
% The argument of this command is the handler identifier given
% by the OSEHRA Journal to this paper.
%
\OTJhandlefooter{\OTJhandlerIDnumber}


\ifpdf
\else
   %
   % Commands for including Graphics when using latex
   %
   \DeclareGraphicsExtensions{.eps,.jpg,.gif,.tiff,.bmp,.png}
   \DeclareGraphicsRule{.jpg}{eps}{.jpg.bb}{`convert #1 eps:-}
   \DeclareGraphicsRule{.gif}{eps}{.gif.bb}{`convert #1 eps:-}
   \DeclareGraphicsRule{.tiff}{eps}{.tiff.bb}{`convert #1 eps:-}
   \DeclareGraphicsRule{.bmp}{eps}{.bmp.bb}{`convert #1 eps:-}
   \DeclareGraphicsRule{.png}{eps}{.png.bb}{`convert #1 eps:-}
\fi


\maketitle


\ifhtml
\chapter*{Front Matter\label{front}}
\fi


\begin{abstract}
\noindent
This article describes the implementation of the Appointment Reminder Card
Program. This program is designed to convert the existing appointment letters
generated by the VISTA Scheduling Package to a more user friendly reminder card
that is easy to understand and provides for an improved patient experience with
the VA health care System.
\begin{center}
\url{http://code.osehra.org/journal/}
\end{center}

\end{abstract}

\tableofcontents

\section{Introduction}

This project is designed to convert the existing appointment letters generated from the VISTA
Scheduling System to a post card sized Appointment Reminder Card that is user friendly, easy to
understand, provides a professional appearance, and reduces cost (production and postage cost).
The project began years ago as a local initiative by Northern California Health Care System (NCHCS)
and transformed to a VISN wide project when it was discovered the return on investment (ROI)
analyses did not support a single facility investing in the equipment resources necessary to generate
the postage sized appointment cards. With the volume of the entire VISN, the ROI demonstrated
significant savings with the reduction of postage, printing, and FTEE cost.


The design of the Appointment Reminder Cards was accomplished by a group of VA staff working
along side a design team under contract with Xerox. The workgroup met with multiple face-to-face
design meetings, and follow-up conference calls and email reviews. Included on the VA Design Team
were representatives from all VISN 21 facilities and staff from VISN 18, with a wide variety of
backgrounds. The team identified 15 types / categories of communication that would replace the over
300 variation of appointment letters used by VISN 21 facilities.

\section{Additional Documentation}

A complete set of documents is available along with this article. They include

\begin{itemize}
\item Appt Card v3 Installation.doc
\item Creation of CLASS definition for R1AC CSP files.doc
\item Importing  R1AC CSP files.doc
\item Part 1 - Appointment Card Implementation Manual Version 1.1 Part 1 of 4.pdf
\item Part 2 - Appointment Card Implementation Manual Version 1.1 Part 2 of 4.pdf
\item Part 3 - Appointment Card Implementation Manual Version 1.1 Part 3 of 4.pdf
\item Part 4 - Appointment Card Implementation Manual Version 1.1 Part 4 of 4.pdf
\item R1AC PCARD GATEWAY SETUP.doc
\item Region 1 Appointment Card SSH setup.doc
\item Xerox Postcard Application setup.doc
\end{itemize}

Please refer to these additional documents for instructions on how to install, use and maintain this package.

%%%%%%%%%%%%%%%%%%%%%%%%%%%%%%%%%%%%%%%%%
%
%  Insert the bibliography using BibTeX
%
%%%%%%%%%%%%%%%%%%%%%%%%%%%%%%%%%%%%%%%%%

\bibliographystyle{plain}
\bibliography{OSEHRAJournal}


\end{document}
