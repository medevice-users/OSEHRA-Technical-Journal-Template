\documentclass{OSEHRAArticle}

\usepackage[dvips]{graphicx}
\usepackage{color}
\usepackage{listings}
\usepackage{verbatim}
\usepackage{textcomp}

\definecolor{listcomment}{rgb}{0.0,0.5,0.0}
\definecolor{listkeyword}{rgb}{0.0,0.0,0.5}
\definecolor{listnumbers}{gray}{0.65}
\definecolor{listlightgray}{gray}{0.955}
\definecolor{listwhite}{gray}{1.0}


%%%%%%%%%%%%%%%%%%%%%%%%%%%%%%%%%%%%%%%%%%%%%%%%%%%%%%%%%%%%%%%%%%
%
%  hyperref should be the last package to be loaded.
%
%%%%%%%%%%%%%%%%%%%%%%%%%%%%%%%%%%%%%%%%%%%%%%%%%%%%%%%%%%%%%%%%%%
\usepackage[dvips,
bookmarks,
bookmarksopen,
backref,
colorlinks,linkcolor={blue},citecolor={blue},urlcolor={blue},
]{hyperref}


\newcommand{\lstlistingwithnumber}[3]{
\begin{center}
\lstinputlisting[linerange={#1-#2},firstnumber=#1,numbers=left]{#3}
\end{center}
}


\title{CPRS}


%
% NOTE: This is the last number of the "handle" URL that
% The OSEHRA Journal assigns to your paper as part of the
% submission process. Please replace the number "1" with
% the actual handle number that you get assigned.
%
\newcommand{\OTJhandlerIDnumber}{2}

\lstset{frame = tb,
       framerule = 0.25pt,
       float,
       fontadjust,
       backgroundcolor={\color{listlightgray}},
       basicstyle = {\ttfamily\footnotesize},
       keywordstyle = {\ttfamily\color{listkeyword}\textbf},
       identifierstyle = {\ttfamily},
       commentstyle = {\ttfamily\color{listcomment}\textit},
       stringstyle = {\ttfamily},
       showstringspaces = false,
       showtabs = false,
       numbers = left,
       numbersep = 6pt,
       numberstyle={\ttfamily\color{listnumbers}},
       tabsize = 2,
       language=[ANSI]C++,
       floatplacement=!h
       }

\release{1.10}

\author{}

\begin{document}


%
% Add hyperlink to the web location and license of the paper.
% The argument of this command is the handler identifier given
% by the OSEHRA Journal to this paper.
%
\OTJhandlefooter{\OTJhandlerIDnumber}


\ifpdf
\else
   %
   % Commands for including Graphics when using latex
   %
   \DeclareGraphicsExtensions{.eps,.jpg,.gif,.tiff,.bmp,.png}
   \DeclareGraphicsRule{.jpg}{eps}{.jpg.bb}{`convert #1 eps:-}
   \DeclareGraphicsRule{.gif}{eps}{.gif.bb}{`convert #1 eps:-}
   \DeclareGraphicsRule{.tiff}{eps}{.tiff.bb}{`convert #1 eps:-}
   \DeclareGraphicsRule{.bmp}{eps}{.bmp.bb}{`convert #1 eps:-}
   \DeclareGraphicsRule{.png}{eps}{.png.bb}{`convert #1 eps:-}
\fi


\maketitle


\ifhtml
\chapter*{Front Matter\label{front}}
\fi


\begin{abstract}
\noindent
This paper presents the Computerized Patient Record System (CPRS), that
provides a single interface for health care providers to review and update a
patient’s medical record, and to place orders, including medications, special
procedures, x-rays, patient care nursing orders, diets, and laboratory tests.
CPRS is flexible enough to be implemented in a wide variety of settings for a
broad spectrum of health care workers, and provides a consistent, event-driven,
Windows-style interface.
\end{abstract}

\tableofcontents

\section{Introduction}

The Computerized Patient Record System (CPRS) is a graphical user interface
intended for clinicians in 1997. CPRS provides a single interface for health
care providers to review and update a patient’s medical record, and to place
orders, including medications, special procedures, x-rays, patient care nursing
orders, diets, and laboratory tests. CPRS is flexible enough to be implemented
in a wide variety of settings for a broad spectrum of health care workers, and
provides a consistent, event-driven, Windows-style interface.

CPRS organizes and presents all relevant data on a patient in a way that
directly supports clinical decision-making. The comprehensive cover sheet
displays timely, patient-centric information, including active problems,
allergies, current medications, recent laboratory results, vital signs,
hospitalization, and outpatient clinic history. This information is displayed
immediately when a patient is selected, and provides an accurate overview of
the patient’s current status before clinical interventions are ordered. CPRS
capabilities include:

A Real-Time Order Checking System that alerts clinicians during the ordering
session that a possible problem could exist if the order is processed; A
Notification System that immediately alerts clinicians about clinically
significant events;

A Patient Posting System, displayed on every CPRS screen, that alerts
clinicians to issues related specifically to the patient, including crisis
notes, warning, adverse reactions, and advance directives; The Clinical
Reminder System, which allows caregivers to track and improve preventive health
care for patients and ensure timely clinical interventions are initiated; and


Remote Data View functionality that allows clinicians to view a patient’s
medical history from other VA facilities to ensure the clinician has access to
all clinically relevant data available at VA facilities.

\section{Conclusions}



%%%%%%%%%%%%%%%%%%%%%%%%%%%%%%%%%%%%%%%%%
%
%  Insert the bibliography using BibTeX
%
%%%%%%%%%%%%%%%%%%%%%%%%%%%%%%%%%%%%%%%%%

\bibliographystyle{plain}
\bibliography{OSEHRAJournal}


\end{document}
